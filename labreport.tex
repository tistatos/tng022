\documentclass[12pt,a4paper]{article}
\pdfoutput=1

%% template from Klas Eskilson
%% This huge block of usepackages does a lot of things.
%% It makes the text pritty on screens and ensures that you can use some common
%% tools.
%%

\usepackage[utf8]{inputenc}
\usepackage[T1]{fontenc}
% change language to whatever you write in
\usepackage[english]{babel}
\usepackage{amsmath}
\usepackage{lmodern}
\usepackage{listings}
\usepackage{units}
\usepackage{icomma}
\usepackage{color}
\usepackage{graphicx}
\usepackage{multicol,caption}
\usepackage{bbm}
\usepackage{hyperref}
\usepackage{xfrac}
\usepackage{lipsum}
\newcommand{\N}{\ensuremath{\mathbbm{N}}}
\newcommand{\Z}{\ensuremath{\mathbbm{Z}}}
\newcommand{\Q}{\ensuremath{\mathbbm{Q}}}
\newcommand{\R}{\ensuremath{\mathbbm{R}}}
\newcommand{\C}{\ensuremath{\mathbbm{C}}}
\newcommand{\rd}{\ensuremath{\mathrm{d}}}
\newcommand{\id}{\ensuremath{\,\rd}}

%  if you want a page header/footer, use this!
% \usepackage{fancyhdr}
% \pagestyle{fancy}
% \lhead{Left in the header}
% \rhead{Right in the header: \today}

% This creates a nice figure environment that puts the image where you use it,
% and not where LaTeX wants it to be.
\newenvironment{Figure}
  {\par\medskip\noindent\minipage{\linewidth}}
  {\endminipage\par\medskip}

% neat horisontal line
\newcommand{\HRule}{\rule{\linewidth}{0.5mm}}

\begin{document}

\title{The title}
\author{The author}
\date{12-12-12}

%%
%% Use one of these title methods
%%
% \maketitle % simpe latex title
\begin{titlepage}
\begin{center}

% Pre-title
\textsc{TNG022 - Modelling and simulation }\\[0.5cm]

% Title
\HRule \\[0.4cm]
{ \huge \bfseries Modelling and simulating a robot for measuring \\[0.4cm] }

\HRule \\[1.5cm]

% Author and supervisor
\begin{minipage}{0.4\textwidth}
\begin{flushleft} \large
% Author
Group 33\\
Carl Englund\\
\emph{caren083@student.liu.se}
Erik Sandrén\\
\emph{erila135@student.liu.se}
\end{flushleft}
\end{minipage}
\begin{minipage}{0.4\textwidth}
\begin{flushright} \large
% supervisor
Shelley Torgnyson\\
\emph{supervisor@example.com}
\end{flushright}
\end{minipage}

\vfill

% Bottom of the page
{\large Laboration performed: November 27, 2014\\}
{\large Report submitted: \today}

\end{center}
\end{titlepage}

% empty page after title page, ignore this page in the numbering
\newpage\null\thispagestyle{empty}\pagenumbering{gobble}\newpage

\newpage\pagenumbering{arabic} % Arabic page numbers (and reset to 1)

% optional abstract!
\begin{abstract}
\lipsum[4]
\end{abstract}

\newpage

% fairly self-explenatory
\tableofcontents

\newpage

\section{Introduction}
The purpose of the laboration is to measure the time constant for a DC-motor with two different software packages. The result of the measurements are compared with each other and also with the time constant given by the data sheet for the DC-motor.

\section{Method and materials}

The DC-motor that was used in the laboration was part of a robot arm used for measuring. A schematic picture of the motor was given, see figure \ref{}.
The two software packages that were used during the laboration was SIMULINK and 20SIM. The purpose of these softwares are to model electrical and mechanical systems. The softwares can also be used to simulate the models.

As a preparation task for the lab the DC-motor was analyzed and a bond graph was constructed. From the bond graph all the numerical constants were determined and a simple block diagram was constructed for the entire motor.

\lipsum[4]

\section{Results}
\lipsum[4]

\section{Discussion}
Discussion about mechanical time constant data sheet gives whole system while the model can give us the mechanical time constant.

Compare results between simulink and 20sim with data sheet.
\lipsum[4]

\section{Conclusion}
\lipsum[4]

% optional bibliography
\begin{thebibliography}{99}

\end{thebibliography}

\end{document}